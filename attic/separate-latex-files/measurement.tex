Since 2010, it has been recognized that accurate cosmography with
individual lens systems involves the following key analysis steps.

\begin{description}
    \item{\bf Time delay estimation} The light curve extracted from
    monitoring observations is used as input to an inference of the
    time delay between the multiple images.
    \item{\bf Lens galaxy mass modeling} High resolution imaging
    and spectroscopic data are used to
    constrain a model for the lens galaxy mass distribution, which can be used
    to predict Fermat potential differences. Both the Einstein ring
    image and the stellar velocity dispersion are important.
    \item{\bf Environment and line of sight modeling} Additional observational
    information about the field of view around the lens system is used
    to account for the weak lensing effects due to massive structures in
    the lens plane and along the line of sight.
\end{description}

Cosmological parameter inference can then proceed -- although in
practice the  separation between this final step and the ones above is
not clean. Practitioners aspire to a joint inference of lens, source,
environment and cosmological parameters from all the data
simultaneously, but have to date broken the problem down  into the above
steps. In the next three sections we describe the current state of the art,
limitations, and principal sources of systematic error of these three
key measurement parts of the problem.
