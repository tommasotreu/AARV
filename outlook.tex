Motivation.

% % % % % % % % % % % % % % % % % % % % % % % % % % % % % % % % % % % %

% \subsection{Cosmic complementarity [TT]}

What's the point? Arent' other probes already doing it? Our place in the
cosmology ecosystem. Discuss place relative to other distance indicators
like Cepheids, BAO, Sne. Then complementarity with growth of structure
probes like weak lensing, clusters etc etc. How important is H0?

Weinberg et al 2013, Kim et al 2014.

Importance of multiple INDEPENDENT measurements for discovery of new
physics.

Refer back to current constraints from \citet{Suy++13} in Section~\ref{sec:cospars}.
Linder SL + SNe.


% % % % % % % % % % % % % % % % % % % % % % % % % % % % % % % % % %

\subsection{Precision [PJM}

Raw precision from Coe \& Moustakas 2009.
Discuss: Sample size. Stage 3, stage 4 surveys. Monitoring solutions.

Extrapolations to N lenses assuming X\% precision per time delay
distance, forecasts.

FIGURE: Forecasts for 10,50,100,1000 lenses for various cosmological
models (w, wa+w0, curvature etc etc). CosmoSIS forecasts (ackn. Dave \&
Elise, ask them).

Note on approximation of forecasting using just Ddt,
and not DA as well: above plot is conservative.
Cite Jee et al paper 1 for pointing this out,
show figure from Jee et al paper 2?


% % % % % % % % % % % % % % % % % % % % % % % % % % % % % % % % % %

\subsection{Accuracy [PJM]}

Discussion of systematic uncertainties:

1) Time delay measurement. Light curve quality.

2) Lens mass modeling. Percent-level systematics due to model
assumptions (ie MSD). IFU observations, resolved stellar kinematics.
Ensembles.

3) Environment and line of sight

Other things: time delay perturbations (someone's noise is somebody else's signal..)
The importance of blinding.
