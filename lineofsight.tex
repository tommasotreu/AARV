% State of the art: ray-traced cosmological simulations, matched via
% number counts.

% Limitations/systematics: incomplete model: local vs line of sight
% mass,  ignoring multi-plane lensing, external convergence only.

The analysis of B1608$+$656 by \citet{Suy++10} explicitly took into
account the weak lensing effects of external structures.  Such a
correction had been suggested by \citet{Fas++06b}, who identified 4
galaxy groups along the line of sight in a spectroscopic survey of the
B1608$+$656 field, and estimated that they could, if left unaccounted
for, bias any inferred Hubble constant high by around 5\%. Exactly how
to  model this effect has been the topic of a number of papers since
2010: the problem is  how to incorporate our knowledge of where the
galaxies are along the line of sight without introducing additional
bias due to the necessary assumptions about how their (dark) mass is
distributed, and how the rest of the mass budget in the field adds up.

\citet{Suy++10} attempted to solve these problems by comparing the
B1608$+$656 field with a large number of fields with similar  galaxy
number overdensity drawn from the Millennium Simulation, modeling the
line of sight effects with a single external convergence parameter and
accepting a somewhat broad prior distribution for it, in return for not
having to make  strong assumptions about the structure of the galaxy
groups in the field. The external convergence in the simulated fields
was calculated by ray-tracing by \citet{Hil++09}, and the
comparison in galaxy overdensity was enabled by the analysis of galaxy
number counts in archival HST images by \citet{FKW11}, who found that
the B1608$+$656 field was overdense by a factor of two. The resulting
prior PDF for the $\kappa_{\rm ext}$ parameter had median $0.10$ with
the 68\% credible interval spanning 0.05 to 0.18.

Since this initial analysis, a number of improvements have been
suggested and investigated. All have in common the desire to bring more
information  to bear on the problem, in order to increase the precision
(while  continuing to avoid introducing bias). \citet{Gre++13} showed
that weighting the galaxy counts by distance, photometric redshift and
stellar mass  can significantly reduce the uncertainty in $\kappa_{\rm
ext}$, by up to 50\%. \citet{Col++13} claim an additional 30\%
improvement  by including knowledge of the stellar mass to halo mass
relation in galaxies,  and modeling each galaxy halo's contribution to
$\kappa_{\rm ext}$ individually in a  3-D reconstruction of the mass in
the field which is then calibrated to simulations in something like the
high resolution limit of the number counts approach. 







While research into these methods continues, one problem in particular
remains outstanding. The methods that involve calibration to numerical
simulations are dependent on the cosmological parameters assumed in that
simulation,  while all methods involve modeling line of sight structures
at various  distances as part of an evolving universe, whose dynamics
depend on cosmological parameters. We face two options: either treat
these cosmological parameters self-consistently  as hyperparameters in a
joint analysis of the time delays and the lens environments, or
demonstrate that they can be decoupled via various simplifying
assumptions that  introduce sub-dominant systematic error.
