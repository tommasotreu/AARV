The measurement of cosmic distances is central to our understanding of
cosmography, i.e. the description of the geometry and kinematics of
the universe. The discovery of the period luminosity relation for
cepheids led to the realization that the universe is much bigger than
the Milky Way and that it is currently expanding. Relative distance
measurements based on supernova Ia light curves were the turning point
in the discovery of the acceleration of the universe
\cite{Riess:1998p21184,Per++99}.

In the two decades since the discovery of the acceleration of the
universe, distance measurements have improved steadily. For example,
the Hubble constant has now been measured to 3\% precision
\cite{Rie++11,Fre++12} 
while the distance to the last scattering surface of the cosmic
microwave backgrond is now known to 1\% precision [check]. This
precision is more than sufficient for all purposes related to our
understanding of phenomena occurring within the universe, like galaxy
evolution.

In spite of all this progress, the most fundamental question still
remains unanswered. What is causing the acceleration? Is this {\it
dark energy} something akin to Einstein's cosmological constant or is
it a dynamical component? Answering this question from an empirical
standpoint will require further improvements in the precision of
distance measurements \cite{Wei++13}.  Many dedicated experiments are
currently under way or being planned with ths goal in mind.

Precision, however, is not sufficient by itself. In addition to
controlling the known statistical uncertainties, modern day
experiments need to control stystematic errors in order to fullfill
their potential, including the infamous unknow unknowns. The most
direct way to demonstrate {\it accuracy} is to compare independent
measurements with comparable {\it precision}.

Ideally, the comparison between independent measurements should be
carried out blindly, so as to minimize experimenter bias. Two blind
mutually blind measurements agreeing that the equation of state
parameter $w$ is not $-1$ would be a very convincing demonstration
that the dark energy is not the cosmological constant. Conversely, the
significant disagreement of two independent measurements, could open
the door to the discovery of new physics.

In this review we focus on gravitational time delay as a tool for
cosmography.  Gravitational time delays are a natural phenomenon in
general relativity and provide a direct and elegant way to measure
absolute distances out to cosmological redshift. When the line of
sight to a distant source of light is suitably well aligned with an
intervening massive system, multiple images appear to the
observer. The arrival time of the images depends on the interplay of
the geometric and gravitational delays specific to the
configuration. If the emission from the source is variable in time,
the difference in arrival time is measurable, and can be converted
into the so-called time delay distance \DDT, a combination of angular
diameter distances to the deflector and source. \DDT is inversely
proportional to the Hubble Constant H$_0$ and it is more weakly
dependent on other cosmological parameters. The sensitivity to H$_0$
and independence to the local distance ladder method make time delays
a very valuable cosmological tool for precise and accurate
cosmology. As several authors have pointed out
\cite{Lin11,Suy++12,Wei++13}, achieving sub-percent precision and
accuracy on the measurement of the Hubble constant is a powerful
addition to stage III and IV dark energy experiments.

This review is organized as follows. In Section~\ref{sec:intro} we
summarize the history of time delay cosmography up until the turn of
the millennium, in order to give a sense of the early challenges and
how they were overcome. In Section~\ref{sec:theory}, we review the
theoretical foundations of the method, in terms of the gravitational
optics version of Fermat's principle. In Section~\ref{sec:timedelay}
we describe in some detail the elements of a modern time delay
distance measurement, emphasizing recent advances and remaining
challenges. In Section~\ref{sec:cosmo} we elucidate the connection
between time delay distance measurements and cosmological parameters,
discussing complementarity with other cosmological
probes. Section~\ref{sec:outlook} critically examines the future of
the method, discussing prospects for increasing the precision, testing
for accuracy, and synergy with other future probes of dark energy. A
brief summary is given in Section~\ref{sec:summary}. Owing to space
limitations, we could only present a selection of all the beautiful
work that has been published on this topic in the past decades. We
refer the readers to recent
\cite{Bar10,Ell10,Tre10,TMC12,Jackson:2013p30763,Jac15,T+E15} and not-so-recent \cite{B+N92,CSS02,K+S04,Fal05,SKW06}
excellent reviews and textbooks \cite{SEF92} for additional
information and historical context.

FIGURE: CARTOON OF LENSING, FROM SPACE WARPS WEBSITE? [PJM]

FIGURE: H0 AS A FUNCTION OF TIME (A CAUTIONARY TALE) [TT]

%xCite Jackson 2015, Weinberg 2013, etc etc.
