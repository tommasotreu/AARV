We reviewed gravitational time delays as a tool for measuring
cosmological parameters. In addition to giving a brief introduction to
the theoretical underpinnings of the method, we discussed the past
history of the field, before turning to present day accomplishments
and the challenges ahead. The main points of this review can be
summarized as follows:

\begin{enumerate}
\item From a theoretical point of view gravitational time delays are a
clean and well understood probe of cosmic acceleration. Conceptually,
each time delay measurement provides a one step measurement of
absolute distance. The typical redshifts of deflectors and sources
span the range between $z\sim2$ and today, covering the era of cosmic
time during which dark energy rose to prominence.
%
\item Even though the potential cosmological application of strongly
lensed, time-variable sources was recognized as early as 1964, it took
decades for the method to come to practical fruition. Two sets of
challenges have been overcome over the past 15 years. Observationally, the
main challenge has been organizing long term monitoring campaigns and
mustering the range of resources required to constrain accurate mass
models. Theoretically, the main challenge consisted of learning how to
exploit the available information to construct lens models with
realistic estimates of the uncertainties.
%
\item It has been demonstrated through blind measurements that each
individual system can deliver a measurement of absolute distance to
about 6-7\% total uncertainty, given current data quality.   The power of
the method is currently limited by the number of systems with
well-measured time delays and sufficient ancillary data to carry out
detailed modeling ($\lesssim10$ at the time of writing).
%
\item Systematic searches for strongly lensed quasars are under way and
should be able to increase the cosmographic sample size by more than
an order of magnitude in the next decade. With improvements in
follow-up image resolution and spatially resolved spectroscopy as
well, we can aspire to to sub-percent precision in the Hubble constant
by the middle of the next decade.
%
\item Before LSST, dedicated monitoring campaigns will the required to
measure each time delay; LSST can potentially alter the landscape, if
it can deliver hundreds of time delays from the survey data themselves.
%
\item Throughout the next decade a promising strategy to deliver the
available precision will consist of obtaining a range of high quality
follow-up data in order to minimize the uncertainty per system. These
include: high resolution imaging from space or with adaptive optics;
redshifts, stellar velocity dispersions, and spatially resolved
kinematics of the deflectors from the James Webb Space Telescope or
large and extremely large ground based telescopes with adaptive
optics.
%
%\item  In the LSST era it will be possible to employ a mixed strategy
%in which large numbers of systems with relatively scant follow-up data
%can be analyzed with hierarchical models based on priors based on the
%in depth analysis of smaller datasets.
%
\item  At the moment, the method is limited by the small number of lens
systemsavailable. As samples increase, further work will be needed to
understand, quantify and mitigate against potential systematic errors
in the method. Extensive parameter recovery tests on realistic
simulated monitoring, high resolution imaging, spatially resolved
spectroscopy, and field weak lensing and photometric data will be
essential to ensure that systematic errors are kept subdominant, the
precision of the method be realized, and an accurate cosmological
measurement made.
\end{enumerate}


In gravitational time delays we have a theoretically sound,
experimentally competitive, and cost-effective cosmographic tool.
Like every other probe, a lot of hard work will be needed to reach the
sub-percent level of precision and accuracy that is needed to make
progress. This effort seems well motivated, not only by the ultimate
goal of improving our understanding of the fundamental constituents of
the universe, but also by the opportunity to use lensed quasars to
learn about the astrophysics of dark matter
\citep{Metcalf:2005p1203,Xu++09,Veg++14,Nie++14}, active galactic
nuclei \citep{PMK08,Blackburne:2010p6600}, and stars
\citep{Sch++14}.

