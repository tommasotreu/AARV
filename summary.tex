This review provides a succint review of gravitational time delays as
a tool for measuring cosmological parameters. In addition to giving a
brief introduction to theoretical underpinnings of the method, the
review discusses the past history of the field, before turning to
present day accomplishments and the challenges ahead. The main points
of this review can be summarized as follows:

\begin{enumerate}
\item From a theoretical point of view gravitational time delays are a clean and well understood probe of cosmic acceleration. Conceptually each time delay measurement provides a direct one step measurement of absolute distance. The typical redshifts of deflectors and sources span the range between $z\sim2$ and today, covering the cosmic time when dark energy arises to prominence.
\item Even though the potential cosmological application of strongly lensed time variable sources was recognized as early as 1964, it took decades for the method to come to fruition in practice. Two sets of challenges were overcome during the past two decades. Observationally, the main challenge was organizing long term monitoring campaigns and mustering the range of resources required to constrain accurate mass models. Theoretically, the main challenge consisted in learning how to exploit the available information to construct lens models with realistic estimate of the uncertainties.
\item Currently, it has been demonstrated through blind measurements that each individual system can deliver a measurement of absolute distance to about 6-7\%total uncertainty.  The power of the method is currently limited by the number of systems with well measured time delays and sufficient ancillary data to carry out detailed modeling ($\lesssim10$ at the time of this writing).
\item Two main ways forward are being pursued. On the one hand, systematic searches for strongly lensed quasars are under way and should be able to increase by 1-2 orders of magnitude the numbers of known suitable systems in the coming decade. With a relatively modest investment of follow-up telescope time, those samples will be sufficient to deliver measurements of cosmic acceleration that are competitive and highly complementary with other probes.
\end{enumerate}


it has been hard to put in practice for two reasons: multi-component
data and logistic considerations.

-Much progress has been made since 2000. Current precision. High
complementary with other probes

-Roadmap


Wise prophesy.
