The measurement of gravitational time delays involves two steps:  taking
monitoring observations of the system over a period of several years,
and then inferring the time delays between the multiple images from
these data.

%   %   %   %   %   %   %   %   %   %   %   %   %   %   %   %   %   %

\subsubsection{Monitoring Observations}

AGN show intrinsic time variability on many scales, with the variability
amplitude increasing  with timescale. Long monitoring campaigns can
build up high statistical  significance as more and more light curve
features can be brought into play.  However, such long campaigns are
difficult to carry out in practice,  because a large number of
guaranteed observing nights are required (even if the total exposure
time is modest). Scheduling such a program has proved difficult, due to
the  competing demands of the rest of the astronomy community. The
highest precision  time delays have come from monitoring campaigns
carried out with dedicated facilities, that is, observatories that were
either able to commit to the long term monitoring proposal submitted, or
that were actually operated in part by the monitoring collaboration.



%State of the art: COSMOGRAIL optical monitoring.

%Systematic errors: uniform calibration and photometry.

Fassnacht for B1608. Long VLA program while Fassnacht was resident.
Well calibrated light curves in radio. Problem is lack of variability.
B1608 yielded good light curve; a few others too. Why only B1608 for cosmology?

COSMOGRAIL. Dedicated network of small telescopes, operated by collaboration.
Necessary focus on bright optical quasars. Achieved high qualuty calibration across
observatories. Low image quality accounted for by deconvolution techniques.

Others?

%   %   %   %   %   %   %   %   %   %   %   %   %   %   %   %   %   %

\subsubsection{Lightcurve Analysis}

%State of the art: TDC results.

%Systematic errors: microlensing, correlated noise.

COSMOGRAIL
TDC
