The measurement of gravitational time delays involves two steps:  taking
monitoring observations of the system over a period of several years,
and then inferring the time delays between the multiple images from
these data.

%   %   %   %   %   %   %   %   %   %   %   %   %   %   %   %   %   %

\subsubsection{Monitoring Observations and Results}

AGN show intrinsic time variability on many scales, with the variability
amplitude increasing  with timescale. Long monitoring campaigns can
build up high statistical  significance as more and more light curve
features can be brought into play.  However, such long campaigns are
difficult to carry out in practice,  because a large number of
guaranteed observing nights are required (even if the total exposure
time is modest). Scheduling such a program has proved difficult, due to
the  competing demands of the rest of the astronomy community. The
highest precision  time delays have come from monitoring campaigns
carried out with dedicated facilities, that is, observatories that were
either able to commit to the long term monitoring proposal submitted, or
that were actually operated in part by the monitoring collaboration.

% State of the art: COSMOGRAIL optical monitoring.

% Systematic errors: uniform calibration and photometry.

Monitoring of the CLASS lens B1608$+$656 in the radio with the Very
Large Array enabled the breakthrough  time delay measurements of
\citep{Fas++02}. In its first season, this program  yielded measurements
of all three time delays in this quadruple image system with  precision
of 6--10\% \citep{Fas++99}; with the variability of the source
increasing over the subsequent two seasons, \citep{Fas++02} were able to
reduce this uncertainty to 2--5\%. Such high precision was the result of
a dedicated campaign during Fassnacht's residence at NRAO, and which
consisted of 8-month seasons, with a mean observation spacing of around
3 days. The light curves were calibrated to 0.6\% accuracy.

While time delays had previously been measured in ten other lens
systems, this was the first time that all  the delays in a quad had been
obtained; moreover, it brought the time  delay uncertainty below the
systematic uncertainty due to the lens model,  prompting new efforts in
this direction beyond what \citep{K+F99} were able to do.

While B1608$+$656 is not the only radio lens with measured time delays,
a combination of factors  led the observational focus to shift towards
monitoring in the optical. With the sample of known, bright lensed
quasars increasing in size,  networks of 1-2m class optical telescopes
began to be investigated. The variability in these systems is somewhat
more reliable, and while microlensing  and image resolution present
observational challenges, the access to data was found to be less
restrictive. The COSMOGRAIL project took on the task of measuring lens
time delays with few-percent precision in this way:  \citet{Eig++05}
showed that microlensing was likely not to be an insurmountable task,
and \citet{Vui++05} provided the proof of  concept with a 4\% precision
time delay measurement in SDSS\ J1650$+$4251.

One of the keys to the success of this program has been the simultaneous
deconvolution of the individual frames in the imaging dataset, using  a
mixture model to describe the point-like quasar images and extended lens
and AGN host galaxies \citep{MCS98}.  Another is the dedicated nature of
the network of telescopes employed, and the  careful calibration of the
photometry across this distributed system. The COSMOGRAIL team have now
published high precision time delays in WFI\,J2033$-$4723
\citep[][3.8\%]{Vui++08}, HE\,0435$-$1223 \citep[][5.6\%]{Cou++11},
SDSS\,J1206$+$4332 \citep[][2.7\%]{Eul++13} and  RX\,J1131$-$1231
\citep[][1.5\%]{Tew++13}, and SDSS\,J1001$+$5027
\citep[][2.8\%]{RK++13}, with more due to follow.  Typically multiple
years of monitoring is needed to obtain an accurate  time delay, as the
variability fluctuates and the reliability of the  measurement converges
\citep[see the discussion in e.g.\ ][]{Tew++13}.


%   %   %   %   %   %   %   %   %   %   %   %   %   %   %   %   %   %

\subsubsection{Lightcurve Analysis Methods}

% State of the art: TDC results.

% Systematic errors: microlensing, correlated noise.

COSMOGRAIL

TDC
